\section{Zespół}
Członkowie zespołu tworzącego aplikację ze wstępnym podziałem na role:
\begin{itemize}
    \item Maja Bojarska:
        \begin{itemize}
            \item Projekt, implementacja i testowanie encji bazy danych wraz z mapowaniem obiektowo-relacyjnym.
            \item Projekt, implementacja i testowanie RESTful API.
        \end{itemize}
    \item Damian Koper:
        \begin{itemize}
            \item Projekt odwzorowania obiektów świata rzeczywistego za pomocą encji w bazie danych.
            \item Projekt i implementacja interfejsu użytkownika.
            \item Testy jednostkowe i integracyjne poszczególnych komponentów systemu.
            \item Testy end-to-end interfejsu użytkownika.
        \end{itemize}
\end{itemize}

\section{Opis systemu}
Gospodarstwo domowe jest miejscem zrzeszającym domowników. Każda z osób ma wkład w jego rozwój. W celu śledzenia i równoważenia kosztów wynikających z użytkowania dostępnych zasobów potrzebny jest system śledzący i obliczający wydatki. Domownicy mogą dzielić się kosztami utrzymania po równo, lub odgórnie zdefiniować zasady określające procentowy wkład każdej osoby.

Działania te, razem ze śledzeniem zużycia mediów (woda, prąd, gaz), pozwolą na dokładniejsze przeanalizowanie wydatków. W przyszłości może się to przełożyć na bardziej świadome zarządzanie zasobami, a dzięki temu na optymalizację ponoszonych kosztów.

\newpage

\section{Wymagania funkcjonalne}

\subsection{Zarządzanie użytkownikami i gospodarstwem}

\begin{enumerate}
    \item Użytkownik może zalogować się na swoje konto.
    \item Użytkownik może wylogować się ze swojego konta.
    \item Użytkownik może edytować dane swojego konta.
    \item Administrator może utworzyć konto użytkownika.
    \item Administrator może edytować dane i rolę użytkownika.
    \item Administrator może usunąć konto użytkownika.
    \item Administrator może zmienić ustawienia procentowego udziału danego użytkownika w wydatkach gospodarstwa.
\end{enumerate}
    
\subsection{Zakupy}

\begin{enumerate}
    \item Użytkownik może wprowadzić dane zakupu przedmiotu.
    \item Użytkownik może edytować dane zakupu przedmiotu.
    \item Użytkownik może usuwać dane zakupu przedmiotu.
    \item Użytkownik może tworzyć sklepy.
    \item Użytkownik może edytować sklepy.
    \item Użytkownik może usuwać sklepy.
    \item Użytkownik może tworzyć kategorie.
    \item Użytkownik może edytować kategorie.
    \item Użytkownik może usuwać kategorie.
\end{enumerate}

\subsection{Listy zakupów}

\begin{enumerate}
    \item Użytkownik może dodać listy zakupów
    \item Użytkownik może edytować listy zakupów
    \item Użytkownik może usuwać listy zakupów
    \item Użytkownik może dodać pozycję listy zakupów.
    \item Użytkownik może edytować pozycję listy zakupów.
    \item Użytkownik może usuwać pozycję listy zakupów.
    \item Użytkownik może przekonwertować listę zakupów na wstępnie utworzone dane o zakupach.
\end{enumerate}

\subsection{Wydatki gospodarstwa}

\begin{enumerate}
    \item Uzytkownik może wprowadzić dane o jednorazowych wydatkach.
    \item Uzytkownik może edytować dane o jednorazowych wydatkach.
    \item Uzytkownik może usuwać dane o jednorazowych wydatkach.
    \item Użytkownik może wprowadzić reguły obliczeń kosztów zużycia mediów.    
    \item Użytkownik może edytować reguły obliczeń kosztów zużycia mediów.    
    \item Użytkownik może usuwać reguły obliczeń kosztów zużycia mediów.    
    \item Użytkownik może wprowadzić dane zużycia mediów.
    \item Użytkownik może edytować dane zużycia mediów.
    \item Użytkownik może usuwać dane zużycia mediów.
\end{enumerate}

\newpage

\subsection{Statystyki i raporty}

\begin{enumerate}
    \item Użytkownik może wyświetlić statystyki kosztów zakupów dla każdego użytkownika, dla ustawionego okresu:
    \begin{enumerate}
        \item Zakupy w czasie jako wykres liniowy.
        \item Zakupy zgrupowane w kategorie jako wykres kolumnowy.
        \item Procentowy udział kategorii we wszystkich zakupach jako wykres kołowy.
    \end{enumerate}
    \item Użytkownik może wyświetlić statystyki dla wszystkich zakupów lub tylko dla zakupów współdzielonych.
    \item Użytkownik może wyświetlić liczbowe podsumowanie danego miesiąca zawierające:
    \begin{enumerate}
        \item Kwoty zakupów współdzielonych podzielone na wszystkich użytkowników z uwzględnionym ustawionym podziałem.
        \item Kwoty do zapłaty dla poszczególnych użytkowników pozostałym użytkownikom wynikające z wprowadzonych zakupów i opłaconych rachunków.
    \end{enumerate}
\end{enumerate}

\section{Wymagania niefunkcjonalne}
\begin{enumerate}
    \item Aplikacja zapewnia bezpieczeństwo sesji użytkownika.
    \item Aplikacja jest odporna na popularne ataki - SQL Injection, XSS, Man in the middle.
    \item Interfejs graficzny aplikacji działa po stronie przeglądarki użytkownika i komunikuje się z jej serwerem za pomocą RESTful API.
    \item Aplikacja zapewnia spójność interfejsu z aplikacjami mobilnymi używając stylu \textit{Material Design}.
    \item Aplikacja zapewnia funkcjonalności Progressive Web App.
    \item Architektura aplikacji musi umożliwiać szybką instalację wszystkich jej komponentów, serwisów i jej uruchomienie za pomocą jednej komendy.
\end{enumerate}

\newpage
\section{Przypadki użycia}

\begin{figure}[!htb]
    \centering
    \includegraphics[keepaspectratio,
    width=\linewidth]{diagrams/out/ucd_user_management.png}
    \caption{Diagram przypadków użycia - użytkownicy.}
\end{figure}

\begin{figure}[!htb]
    \centering
    \begin{minipage}[b]{0.45\textwidth}
        \includegraphics[width=\textwidth]{diagrams/out/ucd_shop.png}
        \caption{Diagram przypadków użycia - sklepy.}
    \end{minipage}
    \hfill
    \begin{minipage}[b]{0.45\textwidth}
        \includegraphics[width=\textwidth]{diagrams/out/ucd_category.png}
        \caption{Diagram przypadków użycia - kategorie.}
    \end{minipage}
\end{figure}

\begin{figure}[!htb]
    \centering
    \includegraphics[keepaspectratio,
    width=\linewidth]{diagrams/out/ucd_shopping_list.png}
    \caption{Diagram przypadków użycia - lista zakupów.}
\end{figure}

\begin{figure}[!htb]
    \centering
    \begin{minipage}[b]{0.45\textwidth}
        \includegraphics[keepaspectratio, width=\textwidth, height=10cm]{diagrams/out/ucd_purchase.png}
        \caption{Diagram przypadków użycia - zakupy.}
    \end{minipage}
    \hfill
    \begin{minipage}[b]{0.45\textwidth}
        \includegraphics[width=\textwidth]{diagrams/out/ucd_stats_reports.png}
        \caption{Diagram przypadków użycia - statystyki.}
    \end{minipage}
\end{figure}
\clearpage
\begin{figure}[!htb]
    \centering
    \includegraphics[keepaspectratio,
    width=\linewidth,
    height=\dimexpr\textheight-2\baselineskip]{diagrams/out/ucd_expenses.png}
    \caption{Diagram przypadków użycia - wydatki.}
\end{figure}
